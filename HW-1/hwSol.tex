\documentclass{article}
\usepackage[utf8]{inputenc}
\usepackage{float}

\begin{document}
\title{Homework Assignment \#1 \\ ECGR 5100}
\author{
  Vishakha Dixit \\
  (801265288) \\
  \texttt{vdixit2@uncc.edu}
}
\date{Fall 2022}
\maketitle

\section*{Problems}
\begin{enumerate}
\item   Translate the following English passages into propositions
        and logical connectors.
        \begin{enumerate}
        \item   {\sffamily ``You can see the movie only if you are
                over 18 years old or you have permission of a parent.''}
                Use $m$, $e$, and $p$ as the symbols.
        \item   {\sffamily ``You can use Linux or you can upgrade your
                Windows operating system.  However, to upgrade Windows
                you need to have a 64-bit processor running at 1 GHz
                or faster, and 16 GB of free hard disk space.''}
        \end{enumerate}

        \textbf{Soln-1 a.} Considering following prepositions, 
        \begin{itemize}
                \item   \textbf{m} = You can see the movie 
                \item   \textbf{e} = You are over 18 years old
                \item   \textbf{p} = you have permission of a parent. 
        \end{itemize}
        Prepositional logic for the above sentence can be written as follows using logical connectors: 
        \begin{center}
       $((e \vee p) \rightarrow m)$
        \end{center}

        \textbf{Soln-1 b.} Considering following prepositions, 
        \begin{itemize}
                \item   \textbf{L} = You can use Linux 
                \item   \textbf{W} = You can upgrade window’s operation system
                \item   \textbf{P} = You have 64-bit processor 
                \item 	\textbf{G} = Your processor runs at 1 GHz or faster
                \item 	\textbf{H} = You have 16 GB free hard disk space
        \end{itemize}
        Prepositional logic for the above sentence can be written as follows using logical connectors: 
        \begin{center}
       $((L \vee (W \leftrightarrow (P \wedge G \wedge H))))$
        \end{center}

\item   Are these system specifications consistent? \par
        {\sffamily ``If the filesystem is not locked, then new messages
        will be queued.  If the filesystem is not locked, then the
        system is functioning normally, and conversely.  If new messages
        are not queued, then they will be sent to the message buffer.
        If the filesystem is not locked, then new messages will be sent
        to the message buffer.  New messages will nto be sent to the
        message buffer.''} \par
        In other words, does this lead to a contradiction?


        \textbf{Soln-2.} A system is considered consistent if assignment 
                        of true value to the variable results in true expression. 
                        For the given system above let’s consider following logical expression: 
        \begin{itemize}
                \item   \textbf{L} = Filesystems are locked.
                \item   \textbf{Q} = New messages will be queued. 
                \item   \textbf{N} = The system is functioning normally. 
                \item 	\textbf{B} = New messages will be sent to message buffer.
        \end{itemize}
        The given specifications can be expressed as follows: 
        \begin{itemize}
                \item   If the file system is not locked, the new messages will be queued: $(\neg L \rightarrow Q)$
                \item   If the file system is not locked, then the system is functioning normally and conversely: $(\neg L \rightarrow N)$
                \item   If new messages are not queued, then they will be sent to the message buffer:  $(\neg  Q \rightarrow B)$
                \item 	If the file system is not locked, then new messages will be sent to the message buffer: $(\neg L \rightarrow B)$
                \item   New messages will not be sent to the message buffer: $(\neg B)$
        \end{itemize}
        To get consistency, take N false in order that $(\neg N)$ be true. This require that both L and Q be true, 
        by the two conditional statements that have N as their consequence. The first conditional statement $(\neg L \rightarrow Q)$ 
        is of the form $(F \rightarrow T)$, which is true. Finally, the $(\neg L \rightarrow B)$ can be satisfied by taking B to be false.
        Therefore, this set of specifications is consistent.

\item   Show that $p \leftrightarrow q$ and
        $(p \wedge q) \vee (\neg p \wedge \neg q)$ are logically equivalent
        using algebra.  (Use the Theorems/Postulates in Rosen's textbook
        or the ones listed in the notes from Harris \& Harris.)
        
        \textbf{Soln-3.}\par
        L.H.S. = $p \leftrightarrow q$\par
           = $(p \rightarrow q) \wedge (q \rightarrow p)$\par
           = $(p \vee \neg q) \wedge (q \vee \neg p)$ [$ because $  $(p \rightarrow q)$ = $(p \vee \neg q)$] \par
           = $((p \vee \neg q) \wedge q) \vee ((p \vee \neg q) \wedge \neg p)$\par
           = $((p \wedge q) \vee (\neg q \wedge q)) \vee ((p \wedge \neg p) \vee (\neg q \wedge \neg p))$ [$ because $  $(\neg Q \wedge Q)$ = 0] \par		
           = $(p \wedge q) \vee (\neg p \wedge \neg q)$\par
        L.H.S. = R.H.S.\par
        \begin{center}
        Q.E.D
        \end{center}
        
\item   Show that $(p \vee q) \wedge (\neg p \vee r) \rightarrow (q \vee r)$
        is a tautology using algebra.

        \textbf{Soln-4.} We know that $L \rightarrow Q  = \neg L \vee Q$ \par
        L.H.S. = $\neg ((p \vee q) \wedge (\neg p \vee r)) \vee (q \vee r)$\par
           = $\neg (p \vee q) \vee \neg (\neg p \vee r) \vee (q \vee r)$ \par
           = $(\neg p \wedge \neg q) \vee (p \wedge \neg r) \vee q \vee r $ \par
           = $(\neg p \wedge \neg q) \vee q \vee (p \wedge \neg r) \vee r$\par
           = $((\neg p \vee q) \wedge (\neg q \vee q)) \vee ((p \vee r) \wedge (\neg r \vee r))$\par
           = $((\neg p \vee q) \wedge 1) \vee ((p \vee r) \wedge 1)$\par
           = $(\neg p \wedge 1) \vee (q \wedge 1) \vee (p \wedge 1) \vee (r \wedge 1)$\par
           = $\neg p \vee q \vee p \vee r$\par
           = $(\neg p \vee p) \vee (q \vee r)$ [$ because $  $(\neg P \vee P)$ = 1] \par
           = 1 $\vee (q \vee r)$
           = 1 	
         
        Therefore it's a tautology.\par
        \begin{center}
        Q.E.D
        \end{center}

\item   For the first four problems, enumerate the truth tables that
        describe the relationships between the propositions.

        \textbf{Soln-5.}  
        \begin{table} [H]
                \centering
                \begin{tabular}{ | c | c | c | }
                \hline
                \textbf{E} & \textbf{P} & \textbf{M} \\ [0.5 ex]
                \hline 
                F & F & F \\
                \hline
                F & T & T \\
                \hline
                T & F & T \\
                \hline
                T & T & T \\ [1ex]
                \hline
                \end{tabular}
                \caption{Truth table for Q-1 (a)}
                \label {table:1}
        \end{table}
        \begin{table} [H]
                \centering
                \begin{tabular}{ | c | c | c | c | c | c | }
                \hline
                \textbf{G} & \textbf{H} & \textbf{L} & \textbf{P} & \textbf{W} & \textbf{$L \vee (W \leftrightarrow (P \wedge G \wedge H))$} \\ [0.5 ex]
                \hline 
                F & F & F & F & F & T \\
                \hline
                F & F & F & F & T & F \\
                \hline
                F & F & F & T & F & T \\
                \hline
                F & F & F & T & T & F \\
                \hline
                F & F & T & F & F & T \\
                \hline
                F & F & T & F & T & T \\
                \hline
                F & F & T & T & F & T \\
                \hline
                F & F & T & T & T & T \\
                \hline
                F & T & F & F & F & T \\
                \hline
                F & T & F & F & T & F \\
                \hline
                F & T & F & T & F & T \\
                \hline
                F & T & F & T & T & F \\
                \hline
                F & T & T & F & F & T \\
                \hline
                F & T & T & F & T & T \\
                \hline
                F & T & T & T & F & T \\
                \hline
                F & T & T & T & T & T \\
                \hline
                T & F & F & F & F & T \\
                \hline
                T & F & F & F & T & F \\
                \hline
                T & F & F & T & F & T \\
                \hline
                T & F & F & T & T & F \\
                \hline
                T & F & T & F & F & T \\
                \hline
                T & F & T & F & T & T \\
                \hline
                T & F & T & T & F & T \\
                \hline
                T & F & T & T & T & T \\
                \hline
                T & T & F & F & F & T \\
                \hline
                T & T & F & F & T & F \\
                \hline
                T & T & F & T & F & F \\
                \hline
                T & T & F & T & T & T \\
                \hline
                T & T & T & F & F & T \\
                \hline
                T & T & T & F & T & T \\
                \hline
                T & T & T & T & F & T \\
                \hline
                T & T & T & T & T & T \\ [1ex] 
                \hline
                \end{tabular}
                \caption{Truth table for Q-1 (b)}
                \label {table:2}
        \end{table}
        \begin{table} [H]
                \centering
                \begin{tabular}{ | c | c | c | }
                \hline
                \textbf{Specification} & \textbf{Condition} & \textbf{Truth Value} \\ [0.5 ex]
                \hline 
                $ \neg L \rightarrow Q$ & $F \rightarrow T$ & T \\
                \hline
                $ \neg L \rightarrow N$ & $F \rightarrow F$ & T \\
                \hline
                $ \neg Q \rightarrow B$ & $F \rightarrow F$ & T \\
                \hline
                $ \neg L \rightarrow B$ & $F \rightarrow F$ & T \\
                \hline
                $ \neg B $ & T & T \\ [1ex]
                \hline
                \end{tabular}
                \caption{Truth table for Q-2}
                \label {table:3}
        \end{table}
        \begin{table} [H]
                \centering
                \begin{tabular}{ | c | c | c | c | c | c | c | c | }
                \hline
                \textbf{p} & \textbf{q} & \textbf{$p \leftrightarrow q$} & \textbf{$p \wedge q$} & \textbf{$ \neg p$} & \textbf{$ \neg q$} & \textbf{$ \neg p \wedge \neg q$} & \textbf{$(p \wedge q) \vee (\neg p \wedge \neg q)$} \\ [0.5 ex]
                \hline 
                F & F & T & F & T & T & T & T \\
                \hline
                F & T & F & F & T & F & F & F \\
                \hline
                T & F & F & F & F & T & F & F \\
                \hline
                T & T & T & T & F & F & T & T \\ [1ex]
                \hline
                \end{tabular}
                \caption{Truth table for Q-3}
                \label {table:4}
        \end{table}
        \begin{table} [H]
                \centering
                \begin{tabular}{ | c | c | c | c | c | c | c | c | }
                \hline
                \textbf{p} & \textbf{q} & \textbf{r} & \textbf{$((p \vee q) \wedge (\neg p \vee r)) \rightarrow (q \vee r)$} \\ [0.5 ex]
                \hline 
                F & F & F & T \\
                \hline
                F & F & T & T \\
                \hline
                F & T & F & T \\
                \hline
                F & T & T & T \\
                \hline
                T & F & F & T \\
                \hline
                T & F & T & T \\
                \hline
                T & T & F & T \\
                \hline
                T & T & T & T \\ [1ex]
                \hline
                \end{tabular}
                \caption{Truth table for Q-4}
                \label {table:5}
        \end{table}

\item   From proposition One: $(p \wedge q) \rightarrow p$, can you infer
        proposition Two: $p \vee q$?  Can you prove this with a truth
        table?  (Do not simply write a truth table, explain how the
        truth table demonstrates the inference.

        \textbf{Soln-6.}Given preposition One: $(p \wedge q) \rightarrow p$. \par
                        We know from De Morgans’s law that $p \rightarrow q$ = $(\neg p \vee q)$ \par
                        $(p \wedge q) \rightarrow p$ = $\neg (p \wedge q) \vee p$ \par
                                     = $\neg p \vee q \vee p$ \par
                                     = $(\neg p \vee p) \vee q$ [$ because $ $(\neg P \vee P)$ = 1] \par
                                     = $1 \vee q$ \par
                                     = 1 [Tautology] \par
        With this given condition we can say that if $(p \wedge q)$ is true then p is true. 
        Also, if p is true then $(p \vee q)$ will always be true. Therefore, we can infer $(p \vee q)$ from $(p \wedge q) \rightarrow p$.

        \begin{table} [H]
                \centering
                \begin{tabular}{ | c | c | c | }
                \hline
                \textbf{p} & \textbf{q} & \textbf{$(p \wedge q) \rightarrow p$} \\ [0.5 ex]
                \hline 
                F & F & T \\
                \hline
                F & T & T \\
                \hline
                T & F & T \\
                \hline
                T & T & T \\ [1ex]
                \hline
                \end{tabular}
                \caption{Truth table for Q-6}
                \label {table:6}
        \end{table}

\item   Use propositional calculus to determine which person did which
        exercise and what beverage they drank after their workout.
        \par
        {\sffamily After an invigorating workout, five fitness-conscious
        friends know that nothing is more refreshing than a tall cool
        glass of mineral water!  Each person (including Annie) has a
        different, favorite form of daily exercise (one likes to
        Rollerblade), and each drinks a different form brand of mineral
        water (one is Crystal).  From the information provided,
        determine the type of exercise and brand of water each person
        prefers.
        \begin{itemize}
        \item   The friends are Annie, Ben, Meg, Page, and Tim.
        \item   The exercises are: \textbf{A}erobics, \textbf{B}icycling,
                \textbf{J}ogging, \textbf{R}ollerblading, and
                \textbf{W}alking.
        \item   The brands of water are Ocean, Crystal, Mountain,
                Purity, and Creek.
        \item   The one who bicycles in pursuit of fitness drinkes Ocean.
        \item   Tim enjoys Aerobics every morning before work.
                Ben is neither the one who drinks Creek nor the one who
                imbibes Ocean.
        \item   Page (who is neither the one who jogs nor the one who
                walks to keep in shape) drinks Purity.
        \item   Meg drinks Mountain but not after jogging.
        \end{itemize}}

        \textbf{Soln-7.} Given that each person has different form of exercises 
                        and different forms of drinks. We consider following logical 
                        expressions for each person, drink, and exercise: \par

                        \textbf{A} = Annie, \textbf{B} = Ben, \textbf{M} = Meg, \textbf{P} = Page, \textbf{T} = Tim \par
                        \textbf{Ae} = Aerobics, \textbf{Bi} = Bicycles, \textbf{Jo} = Jogging, \textbf{Wa} = Walk \par
                        \textbf{Oc} = Ocean, \textbf{Cry} = Crystal, \textbf{Mo} = Mountain, \textbf{Pu} = Purity, \textbf{Cr} = Creek \par

                        Given: \par
                        1. $Bi \rightarrow Oc$ \par
                        2. $T \rightarrow Ae$ \par
                        3. $B \rightarrow \neg (Cr \wedge Oc)$ \par
                        4. $P \rightarrow \neg (Jo \wedge Wa)$ \par
                        5. $P \rightarrow Pu$ \par
                        6. $M \rightarrow Mo$ \par
                        7. $M \rightarrow ¬Jo$ \par
                        If Annie drives bicycles then \par
                        8.  $A \rightarrow Bi$	[Assuming] \par
                        9.  $A \rightarrow Oc$	[8, 1 Hypothetical Syllogism] \par
                        10. $P \rightarrow Ro$	[4 = Page Doesnt Jog and Walk, 8 = Annie rides Bicycle, 2 = Tim does Aerobics] \par
                        11. $M \rightarrow Wa$	[7 = Meg Doesnt Jog, 8, 2, 10] \par
                        12. $B \rightarrow Cry$	[3, 5, 6] \par
                        13. $T \rightarrow Cr$	[5, 6, 9, 12] \par

                        \begin{table} [H]
                                \centering
                                \begin{tabular}{ | c | c | c | c | c | c | c | c | c | c | c | }
                                \hline
                                \textbf{Person} & \textbf{Ae} & \textbf{Bi} & \textbf{Jo} & \textbf{Ro} & \textbf{Wa} & \textbf{Oc} & \textbf{Cry} & \textbf{Mo} & \textbf{Pu} & \textbf{Cr} \\ [0.5 ex]
                                \hline 
                                Annie & F & \textbf{T} & F & F & F & \textbf{T} & F & F & F & F \\
                                \hline
                                Ben & F & F & \textbf{T} & F & F & F & \textbf{T} & F & F & F \\
                                \hline
                                Meg & F & F & F & F & \textbf{T} & F & F & \textbf{T} & F & F \\
                                \hline
                                Page & F & F & F & \textbf{T} & F & F & F & F & \textbf{T} & F \\ 
                                \hline
                                Tim & \textbf{T} & F & F & F & F & F & F & F & F & \textbf{T} \\ [1ex]
                                \hline
                                \end{tabular}
                                \caption{Truth table for Q-7}
                                \label {table:7}
                        \end{table}

\item    Read the journal \emph{Science} news article:
        Write a paragraph summarizing the \emph{implications} of the
        policy change.
        \par
        Write a second paragraph taking a stand for or against the
        change and justify why it is good for science.

        \textbf{Soln-8 a.} Public access policy announced by the US government states that US research agencies 
                        should make the results of federally funded research free to read as soon as they are published. 
                        This is a significant shift that will lift the 1 year subscription paywall and promote accessibility to 
                        research results and will also improve people's trust in science. This implies that US taxpayers will 
                        not have to pay to access results of the research funded by their tax money. On the other hand, the new policy 
                        does not describe how the cost of production and editing these research papers will be funded. Thus, making publishing 
                        more difficult for authors. Another implication is that without subscription revenues open-access business model will 
                        require authors pay a fee to make their papers immediately available for free to the public. 

        \textbf{Soln-8 b.} Free access to federally funded research and data is very important for accelerating discovery, 
                        encouraging collaboration, and ensuring government accountability. The value of these research data can 
                        be observed from the early days of COVID 19 pandemic. Researchers rapidly shared their data through a 
                        wide variety of online resources which not only accelerated discovery but also allowed the translation 
                        of research into prevention strategies, treatments, vaccines, and standards of care that ultimately saved 
                        lives, despite the ongoing toll of the pandemic. Open access policy provides free flow of information that 
                        will benifit underfunded groups of students and researchers who cannot pay high subscription cost. US taxpayers 
                        fund billions of dollars of research every year in various fields of science. A portion of this 
                        grant can be saved to pay the publication cost in pay-to-publish model. Open access policy can be a 
                        big advancement moving forward, as in the long run it will make science data more accesible to public which 
                        may induce trust in science and scientific procedure.

\end{enumerate}
\end{document}